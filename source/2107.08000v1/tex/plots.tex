%------------------------------------------------------------------------------
% hack for CVPR/ICCV style
\makeatletter
\@namedef{ver@everyshi.sty}{}
\makeatother

%------------------------------------------------------------------------------
% main packages
\usepackage[dvipsnames,svgnames,x11names]{xcolor}
\usepackage{tikz}
\usetikzlibrary{arrows.meta,shapes,calc,matrix,fit,positioning,backgrounds,decorations.markings,fadings}
\usepackage{pgfplots}
\usepackage{pgfplotstable}
\pgfplotsset{compat=1.9}
\usepackage{xstring}

%------------------------------------------------------------------------------
% externalization: requires defining \finalcopy first

\usepgfplotslibrary{external}
% \tikzexternalize[prefix=fig/extern/]
\newcommand{\extfig}[2]{\tikzsetnextfilename{#1}{#2}}
\newcommand{\noextfig}[1]{\tikzset{external/export next={false}}{#1}}
\newcommand{\extdata}[1]{\input{#1}}
\IfBeginWith*{\jobname}{fig/extern/}{\finalcopy}{}

%-----------------------------------------------------------------------------
% tikz styles

\tikzstyle{every picture}+=[
	remember picture,
	every text node part/.style={align=center},
	every matrix/.append style={ampersand replacement=\&},
]
\tikzstyle{tight} = [inner sep=0pt,outer sep=0pt]
\tikzstyle{node}  = [draw,circle,tight,minimum size=12pt,anchor=center]
\tikzstyle{op}    = [draw,circle,tight]
\tikzstyle{dot}   = [fill,draw,circle,inner sep=1pt,outer sep=0]
\tikzstyle{pt}    = [fill,draw,circle,inner sep=1.5pt,outer sep=.2pt]
\tikzstyle{box}   = [draw,thick,rectangle,inner sep=3pt]
\tikzstyle{high}  = [black!60]
\tikzstyle{group} = [high,box,opacity=.5]
\tikzstyle{dim1}  = [fill opacity=.3,text opacity=1]
\tikzstyle{dim2}  = [fill opacity=.5,text opacity=1]
\tikzstyle{dim3}  = [fill opacity=.7,text opacity=1]
\tikzstyle{rectc} = [tight,transform shape]
\tikzstyle{rect}  = [rectc,anchor=south west]

%-----------------------------------------------------------------------------
% framed figures

\newcommand{\framed}[3][1]{\extfig{#2}{\tikz{
	\node[tight](a){\fig[#1]{#3}};
	\node[tight,draw=gray,fit=(a)]{};
}}}

%-----------------------------------------------------------------------------
% pgfplots general options

\newcommand{\leg}[1]{\addlegendentry{#1}}

\tikzset{every mark/.append style={solid}}
\pgfplotsset{%smooth,
	grid=both, width=\columnwidth, try min ticks=5,
	every axis/.append style={font=\small},
	every axis plot/.append style={thick,mark=none,mark size=1.8,tension=0.18},
	legend cell align=left, legend style={fill opacity=0.8},
	xticklabel={\pgfmathprintnumber[assume math mode=true]{\tick}},
	yticklabel={\pgfmathprintnumber[assume math mode=true]{\tick}},
	nodes near coords math/.style={
		nodes near coords={\pgfmathprintnumber[assume math mode=true]{\pgfplotspointmeta}},
	},
}

\pgfplotsset{
	dash/.style={mark=o,dashed,opacity=0.6},
	dott/.style={mark=o,dotted,opacity=0.6},
	nolim/.style={enlargelimits=false},
	plain/.style={every axis plot/.append style={},nolim,grid=none},
}
\newcommand{\kilo}[1]{\thisrow{#1}/1000}

%--------------------------------------------------------------------

% layers
\pgfdeclarelayer{bg4}
\pgfdeclarelayer{bg3}
\pgfdeclarelayer{bg2}
\pgfdeclarelayer{bg1}
\pgfdeclarelayer{fg1}
\pgfdeclarelayer{fg2}
\pgfdeclarelayer{fg3}
\pgfdeclarelayer{fg4}
\pgfsetlayers{bg4,bg3,bg2,bg1,main,fg1,fg2,fg3,fg4}

%------------------------------------------------------------------------------
% 3d drawing

\pgfkeys{/tikz/.cd, aspect/.store in=\aspect, aspect=1}
\pgfkeys{/tikz/.cd, depth/.store in=\depth, depth=.5}
\pgfkeys{/tikz/.cd, stepx/.store in=\step, stepx=1}

\tikzstyle{geom} = [line join=bevel,aspect=1,depth=.5,z={(\depth*\aspect,\depth)}]
\tikzstyle{wire} = [geom,draw,thick]

% 3d coordinates
\def\cx[#1,#2,#3]{#1}
\def\cy[#1,#2,#3]{#2}
\def\cz[#1,#2,#3]{#3}
\def\ex[#1,#2,#3]{#1,0,0}
\def\ey[#1,#2,#3]{0,#2,0}
\def\ez[#1,#2,#3]{0,0,#3}

% lines along x, y, or z
\newcommand{\zline}[3][]{%
\path[geom,#1] #2 -- +(\cx[#3],\cy[#3]);
}
\newcommand{\yline}[3][]{%
\path[geom,#1,shift={#2},xslant=\aspect]
	(0,0) -- +(\cx[#3],\depth*\cz[#3]);
}
\newcommand{\xline}[3][]{%
\path[geom,#1,shift={#2},yslant=1/\aspect]
	(0,0) -- +(\aspect*\depth*\cz[#3],\cy[#3]);
}

% rectangles / grids along x, y, or z
\newcommand{\zrect}[3][]{%
\path[geom,#1] #2 rectangle +(\cx[#3],\cy[#3]);
}
\newcommand{\yrect}[3][]{%
\path[geom,#1,shift={#2},xslant=\aspect]
	(0,0) rectangle +(\cx[#3],\depth*\cz[#3]);
}
\newcommand{\xrect}[3][]{%
\path[geom,#1,shift={#2},yslant=1/\aspect]
	(0,0) rectangle +(\aspect*\depth*\cz[#3],\cy[#3]);
}
\newcommand{\xgrid}[3][]{%
\path[geom,#1,shift={#2},yslant=1/\aspect,xstep=\aspect*\depth*\step]
	(0,0) grid +(\aspect*\depth*\cz[#3],\cy[#3]);
}

% parallepiped
\newcommand{\para}[4][]{%
\zrect[#1]{(#3)}{#4}                 % front
\yrect[#1]{($(#3)+(\ey[#4])$)}{#4}   % top
\xrect[#1]{($(#3)+(\ex[#4])$)}{#4}   % right
\path[geom]
	(#3) coordinate(#2-southwest)
	($(#3)+(#4)$) coordinate(#2-northeast)
	($(#3)+(\ey[#4])$) coordinate(#2-northwest)
	($(#3)+(#4)-(\ey[#4])$) coordinate(#2-southeast)
	($(#3)+.5*(\ex[#4])$) coordinate(#2-south)
	($(#3)+(#4)-.5*(\ex[#4])$) coordinate(#2-north)
	($(#3)+.5*(\ex[#4])+.5*(#4)$) coordinate(#2-center)
	(#2-southwest |- #2-center) coordinate(#2-west)
	(#2-center -| #2-northeast) coordinate(#2-east)
	;
}
